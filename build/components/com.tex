% dhgefigure -> ...
\DeclareDocumentCommand{\dhgefigure}{O{h} m m m m O{} O{}}
{
	\begin{figure}[#1]
		\begin{center}
			\includegraphics[#3]{#2}
		\end{center}
		\caption{#4}
        \label{#5}

		\ifx #6\empty \else
			\ifx #7\empty \else
				{\small \protect \textbf{Quelle:} \cite[#7]{#6}}
			\fi
		\fi
	
	\end{figure}
}

\newcommand{\markBox}[2]
{
    \def\default{#2 {$\square$} #2 {$\square$} #2 {$\square$} #2 {$\square$}}

    \ifnum#1 = 1
    \def\default{#2 {$\boxtimes$} #2 {$\square$} #2 {$\square$} #2 {$\square$}}
    \else
        \ifnum#1 = 2
        \def\default{#2 {$\square$} #2 {$\boxtimes$} #2 {$\square$} #2 {$\square$}}
        \else
            \ifnum#1 = 3
            \def\default{#2 {$\square$} #2 {$\square$} #2 {$\boxtimes$} #2 {$\square$}}
            \else
                \ifnum#1 = 4
                \def\default{#2 {$\square$} #2 {$\square$} #2 {$\square$} #2 {$\boxtimes$}}
                \fi
            \fi
        \fi
    \fi
    \hspace*{-.5cm}\default
}

% new command -> \doubleunderline
\newcommand{\doubleunderline}[1]{
	\underline{\underline{#1}}
}

% definiert eine neue Liste für das Anlagenverzeichnis
\newcommand{\listexamplename}{\vspace*{-20pt}}
\newlistof{anlagen}{alt}{\listexamplename}

% Befehl welcher ein Item dem Anlagenverzeichnis hinzufügt
\newcommand{\addtoanlagen}[2]{%
\def\in{#1}
    \def\fig{fig}
    \def\tab{tab}

    \ifx\fig\in
        \def\type{\thefigure}
        \def\name{Abb.\hspace{8pt}}
        \else \ifx\tab\in
            \def\type{\thetable}
            \def\name{Tab.\hspace{10pt}}
        \fi
    \fi

    \setcounter{anlagen}{\type}
    \addcontentsline{alt}{anlagen}
    {\name\protect\numberline{\theanlagen}\quad\nameref{#1:#2}}\par
}
