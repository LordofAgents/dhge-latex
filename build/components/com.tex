% dhgefigure -> ...
\DeclareDocumentCommand{\dhgefigure}{O{h} m m m m O{} O{}}
{
	\begin{figure}[#1]
		\begin{center}
			\includegraphics[#3]{#2}
		\end{center}
		\caption{#4}
		\label{#5}

		\ifx #6\empty \else
			\ifx #7\empty \else
				{\small \protect \textbf{Quelle:} \cite[#7]{#6}}
			\fi
		\fi

	\end{figure}
}

\newcommand{\markBox}[2]
{
	\def\default{#2 {$\square$} #2 {$\square$} #2 {$\square$} #2 {$\square$}}

	\ifnum#1 = 1
		\def\default{#2 {$\boxtimes$} #2 {$\square$} #2 {$\square$} #2 {$\square$}}
	\else
		\ifnum#1 = 2
			\def\default{#2 {$\square$} #2 {$\boxtimes$} #2 {$\square$} #2 {$\square$}}
		\else
			\ifnum#1 = 3
				\def\default{#2 {$\square$} #2 {$\square$} #2 {$\boxtimes$} #2 {$\square$}}
			\else
				\ifnum#1 = 4
					\def\default{#2 {$\square$} #2 {$\square$} #2 {$\square$} #2 {$\boxtimes$}}
				\fi
			\fi
		\fi
	\fi
	\hspace*{-.5cm}\default
}

% new command -> \doubleunderline
\newcommand{\doubleunderline}[1]{
	\underline{\underline{#1}}
}

% definiert eine neue Liste für das Anlagenverzeichnis
\newcommand{\listexamplename}{\vspace*{-20pt}}
\newlistof{anlagen}{alt}{\listexamplename}

% Befehl welcher ein Item dem Anlagenverzeichnis hinzufügt
\newcommand{\ATA}[1]{%
	\def\fig{fig}
	\def\tab{tab}

	\ifx\fig\typeOfCap
		\def\type{\thefigure}
		\def\name{Abb.\hspace{8pt}}
	\else \ifx\tab\typeOfCap
			\def\type{\thetable}
			\def\name{Tab.\hspace{10pt}}
		\fi
	\fi
	\setcounter{anlagen}{\type}

	% only here because the \type-counter is one lower (later it will count up like normal)
	% -> reason... it's called to early but can't called later because of dependencies other types LUL
	% works only in the last section of the paper so it should be fine :)   
	\refstepcounter{anlagen}

	\addcontentsline{alt}{anlagen}
	{\name\protect\numberline{\theanlagen}\quad#1}\par
}

\newcommand{\renewFigTabCap}{
	% caption ruft jetzt \ATA{} auf welches das jeweilige Objekt zum "Anlageverzeichnis" hinzufügt 
	\let\oldCap=\caption
	\renewcommand{\caption}[1]{\ATA{##1}\oldCap{##1}}

	% redefine table and figure -> table and figure set a global variable on the specific value
	\let\oldTab=\table
	\renewcommand{\table}{\def\typeOfCap{tab}\oldTab}

	\let\oldFig=\figure
	\renewcommand{\figure}{\def\typeOfCap{fig}\oldFig}
}

% formatierung der ba teile autorreferat und thesenblatt 
\newcommand{\baFormat}[2]{
	\begin{center}
		{\LARGE\bf #1}

		\vspace{0.7cm}
		{\large\bf\enquote{\CTITLE}}

		\vspace{0.5cm}
		von \CAUTHOR
	\end{center}

	\vspace{1.5cm}

	{#2}

	\cleardoublepage
}
