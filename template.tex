% ROOT & PACKAGE SETUP
\documentclass	[a4paper, 12pt]{article}
\usepackage		[a4paper, inner=3cm, outer=2.5cm, top=2.5cm, bottom=2.5cm]{geometry}
\usepackage		[utf8]{inputenc}
\usepackage		{setspace}
\usepackage		{tocloft}
\usepackage		{titletoc}
\usepackage		[hidelinks]{hyperref}
\usepackage		{amssymb}
\usepackage		{fancyhdr}
\usepackage		{acronym}
\usepackage		{enumitem}
\usepackage		[T1]{fontenc}
\usepackage		[style=german]{csquotes}
\usepackage		[backend=biber, style=alphabetic, citestyle=alphabetic-ibid, giveninits=true, ibidtracker=true]{biblatex} 
				\addbibresource	{bib.bib}
\usepackage		[ngerman]{babel}
\usepackage		{csquotes,xpatch}

% DOCUMENT SETUP
\onehalfspacing
\setlength		{\parindent}{0cm} % Default is 15pt.
\setlength		{\parskip}{0.5em}
\sloppy
\selectlanguage	{ngerman}

% HEADERS & FOOTERS
\pagestyle		{fancyplain}
\fancyfoot		{}
\fancyhf		{}
\renewcommand		{\headrulewidth}{0pt}
\renewcommand		{\footrulewidth}{0pt}
\fancyhead		[c]{\thepage}

% ToC SETUP
\cftsetindents		{section}{0em}{4em}
\cftsetindents		{subsection}{0em}{4em}
\cftsetindents		{subsubsection}{0em}{4em}
\renewcommand		{\contentsname}{Inhaltsverzeichnis}
\setcounter		{tocdepth}{3}
\setcounter		{secnumdepth}{3}

% ABBILDUNGS-& TABELLENVERZEICHNIS SETUP
\renewcommand	{\listfigurename}{Abbildungsverzeichnis}
\renewcommand	{\listtablename}{Tabellenverzeichnis}

% CITATION & BIBLIOGRAPHY SETUP
\newbibmacro*{oko-cite}{%
	\global\boolfalse{cbx:loccit}%
	\ifthenelse{\ifciteibid\AND\NOT\iffirstonpage}{\usebibmacro{cite:ibid}}
	{\printtext[brackets]{%
			\printfield{labelprefix}%
			\printfield{labelalpha}%
			\printfield{extraalpha}%
			\ifbool{bbx:subentry}
			{\printfield{entrysetcount}}
			{}}}
}

\newbibmacro*{cite:ibid}{%
	\printtext[bibhyperref]{\bibstring[\mkibid]{ibidem}}%
	\ifloccit
	{\global\booltrue{cbx:loccit}}
	{}}

\DeclareCiteCommand{\footcite}[\mkbibfootnote]
{\usebibmacro{prenote}}
{\usebibmacro{citeindex}%
	{\usebibmacro{oko-cite}}}
{\multicitedelim}
{\usebibmacro{postnote}}

\DeclareFieldFormat{url}{\newline\url{#1}}
\DeclareFieldFormat
[manual,online,article,inbook,incollection,inproceedings,patent,thesis,unpublished]
{title}{\mkbibquote{#1\isdot}\addcomma}
\DeclareDelimFormat[bib]{nametitledelim}{\addcolon\space}
\DeclareNameAlias{author}{last-first}
\DeclareNameAlias{editor}{last-first}
\renewcommand{\finentrypunct}{}


% CONSTANTS
\newcommand\CTITLE		{THEMA}
\newcommand\CAUTHOR		{AUTHOR}
\newcommand\CDATUM		{DATUM}
\newcommand\CMATRIKEL		{MATRIKELNUMMER}
\newcommand\CKURS		{KURS}
\newcommand\CCAMPUS		{Gera}
\newcommand\CBEREICH		{Technik}
\newcommand\CSTUDIENGANG	{STUDIENGANG}
\newcommand\CBETRIEB		{FIRMA}
\newcommand\CBETREUER		{BETREUER}

% CUSTOM CONSTANTS
	% DEFINE YOUR CUSTOM CONSTANTS HERE

% COUNTER (So we can keep counting in Roman after having switched to Arabic)
\newcounter{savepage}

%OPENING
\title{\large		{\textbf{\CTITLE}}}
\author			{}
\date			{}

\begin{document}
\pagenumbering	{gobble}
\vspace{\fill}
\maketitle

\if\CARBEIT B

	\begin{center}
		{\LARGE\bf Bachelorarbeit}
		
		\vspace{0.5cm}vorgelegt am \CDATUM
	\end{center}

	\vspace{1cm}

	\def\BETREUER{Gutachter}

\else

	\begin{tabular}{rcccc}
		\hspace{0.45\textwidth} &       I       &     II      &     III     &     IV      \\
	{Projektarbeit Nr.}  \markBox{\CARBEIT}{&}
	\end{tabular}

	\begin{tabular}{rl}
		\hspace{0.45\textwidth} &       \\
		vorgelegt am: & \CDATUM
	\end{tabular}

	\def\BETREUER{Betreuer}

\fi

\begin{tabular}{rl}
	\hspace{0.45\textwidth} &              \\
			von: & \CAUTHOR
\end{tabular}

\begin{tabular}{rl}
	\hspace{0.45\textwidth} &         \\
	Matrikelnummer: & \CMATRIKEL
\end{tabular}

\begin{tabular}{rl}
	\hspace{0.45\textwidth} &      \\
	DHGE Campus: & \CCAMPUS
\end{tabular}

\begin{tabular}{rl}
	\hspace{0.45\textwidth} &         \\
	Studienbereich: & \CBEREICH
\end{tabular}

\begin{tabular}{rl}
	\hspace{0.45\textwidth} &                       \\
	Studiengang: & \CSTUDIENGANG
\end{tabular}

\begin{tabular}{rl}
	\hspace{0.45\textwidth} &       \\
			Kurs: & \CKURS
\end{tabular}

\begin{tabular}{rl}
	\hspace{0.45\textwidth} &          \\
	Ausbildungsstätte: & \CBETRIEB
\end{tabular}

\begin{tabular}{rl}
	\hspace{0.45\textwidth} &          \\
	\BETREUER: & \CBETREUER
\end{tabular}

\vspace*{\fill}

\pagebreak


% INHALTSVERZEICHNIS
\pagenumbering{Roman} \setcounter{page}{1}
\tableofcontents{\fancyfoot{}}
\cleardoublepage

% ABBILDUNGSVERZEICHNIS
\phantomsection
\addcontentsline{toc}{section}{\listfigurename}
\listoffigures
\cleardoublepage

% TABELLENVERZEICHNIS
\phantomsection
\addcontentsline{toc}{section}{\listtablename}
\listoftables
\cleardoublepage

% ABKÜRZUNGSVERZEICHNIS
\section*{Abkürzungsverzeichnis}
\begin{acronym}[VISBHO]\itemsep0pt
	% Definieren Sie hier Ihre Abkürzungen anhand des DHGE Beispiels.
% Wenn Sie DHGE dann im Text verwenden, rufen sie einfach \ac{dhge} auf.
% LaTeX kümmert sich um den Rest.
% Für alles Weitere schauen Sie sich bitte die Dokumentation des Acronym Packages an.

\acro	{dhge}	[DHGE]		{Duale Hochschule Gera Eisenach}

\end{acronym}
\addcontentsline{toc}{section}{Abkürzungsverzeichnis}
\cleardoublepage

\setcounter{savepage}{\arabic{page}}

% MAIN CONTENT
\pagenumbering	{arabic}
Testtext\footcite{Xarticle} 
\ac{dhge}
\dhgefigure[h]{img}{scale=0.75}{Ein Testbild}{fig:test}[Xarticle][S. 17ff]
\dhgefigure[h]{img}{scale=0.75}{Ein Testbild}{fig:test}
\doubleunderline{$150\mathrm{\Omega}$}
\listofanlagen
\addtoanlagen{fig}{test}
\cleardoublepage

% LITERATURVERZEICHNIS
% TODO Formatierung
\pagenumbering	{Roman} \setcounter{page}{\thesavepage}
\printbibliography[title=Literaturverzeichnis]
\addcontentsline{toc}{section}{Literaturverzeichnis}
\cleardoublepage

%TODO %ANLAGENVERZEICHNIS UND ANLAGEN
\section*{Anlagenverzeichnis}
\addcontentsline{toc}{section}{Anlagen}
\cleardoublepage
% EHRENWÖRTLICHE ERKLÄRUNG
\pagenumbering	{gobble}
\section*{Ehrenwörtliche Erklärung}
Ich erkläre hiermit ehrenwörtlich,
\begin{enumerate}[leftmargin=0.5cm]
	\item 	dass ich meine Projektarbeit mit dem Thema:  \\
			\textbf{\CTITLE} \\
			ohne fremde Hilfe angefertigt habe, \\
	\item	dass ich die Übernahme wörtlicher Zitate aus der Literatur sowie die Verwendung der
			Gedanken anderer Autoren an den entsprechenden Stellen innerhalb der Arbeit gekennzeichnet habe und  \\
	\item	dass ich meine Projektarbeit/Studienarbeit/Bachelorarbeit bei keiner anderen Prüfung vorgelegt habe. \\\\
			Ich bin mir bewusst, dass eine falsche Erklärung rechtliche Folgen haben wird. \\\\
\end{enumerate}
\vspace*{\fill}
\begin{tabular} {lrl}
	\hspace{6cm} & \hspace{3cm} & \hspace{6cm} \\
	\hrulefill & & \hrulefill \\
	Ort, Datum & & Unterschrift
\end{tabular}
\vspace*{\fill}

\addcontentsline{toc}{section}{Ehrenwörtliche Erklärung}

\end{document}
