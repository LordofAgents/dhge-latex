% ROOT & PACKAGE SETUP
\documentclass	[a4paper, 12pt]{article}
\usepackage		[a4paper, inner=3cm, outer=2.5cm, top=2.5cm, bottom=2.5cm]{geometry}
\usepackage		[utf8]{inputenc}
\usepackage		{lipsum}
\usepackage		{setspace}
\usepackage		{titletoc}
\usepackage		[hyperfootnotes=false, hidelinks]{hyperref}
\usepackage		{amssymb}
\usepackage		{fancyhdr}
\usepackage		{acronym}
\usepackage		{enumitem}
\usepackage		[T1]{fontenc}
\usepackage		[style=german]{csquotes}
\usepackage		[backend=biber, style=alphabetic, citestyle=alphabetic-ibid, giveninits=true, ibidtracker=true]{biblatex} 
				\addbibresource	{bib.bib}
\usepackage		[ngerman]{babel}
\usepackage		{csquotes,xpatch}
\usepackage		[perpage]{footmisc}
\usepackage		{graphicx}
\graphicspath	{ {./img/} }
\usepackage		{caption}
\usepackage		{ifthen}
\usepackage		{xparse}
\usepackage		{float}
\usepackage		{etoolbox}
\usepackage		{tocloft}

% DOCUMENT SETUP
\onehalfspacing
\setlength		{\parindent}{0cm} % Default is 15pt.
\setlength		{\parskip}{0.5em}
\widowpenalty10000
\clubpenalty10000

\selectlanguage	{ngerman}

% HEADERS & FOOTERS
\pagestyle		{fancyplain}
\fancyfoot		{}
\fancyhf		{}
\renewcommand	{\headrulewidth}{0pt}
\renewcommand	{\footrulewidth}{0pt}
\fancyhead		[c]{\thepage}
\setlength		{\headheight}{15pt}

% ToC SETUP
\cftsetindents	{section}{0em}{4em}
\cftsetindents	{subsection}{0em}{4em}
\cftsetindents	{subsubsection}{0em}{4em}
\renewcommand	{\contentsname}{Inhaltsverzeichnis}
\setcounter		{tocdepth}{3}
\setcounter		{secnumdepth}{5}

% ABBILDUNGEN UND TABELLEN SETUP 
\renewcommand	{\listfigurename}{Abbildungsverzeichnis}
\renewcommand	{\listtablename}{Tabellenverzeichnis}

\addto{\captionsngerman}{%
	\renewcommand*{\figurename}{Abb.}
	\renewcommand*{\tablename}{Tab.}
}

\makeatletter
\renewcommand{\cftfigpresnum}{Abb. }
\renewcommand{\cfttabpresnum}{Tab. }

\setlength{\cftfignumwidth}{2cm}
\setlength{\cfttabnumwidth}{2cm}

\setlength{\cftfigindent}{0cm}
\setlength{\cfttabindent}{0cm}
\makeatother

% CAPTION SETUP
\captionsetup{%
	font=small,
	labelfont=bf,
	singlelinecheck=false,
	skip=10pt,
	belowskip=0pt
}

\DeclareDocumentCommand{\dhgefigure}{O{h} m m m O{} O{}}
{
	\begin{figure}[#1]
		\begin{center}
			\includegraphics[#3]{#2}
		\end{center}
		\caption{#4}
	
		\ifx #5\empty \else
			\ifx #6\empty \else
				{\small \protect \textbf{Quelle:} \cite[#6]{#5}}
			\fi
		\fi
	
	\end{figure}
}

% AUTISTIC CONDITIONALS
% This hides ToF and ToT if they are empty
\makeatletter
\AtEndEnvironment	{figure}{\gdef\there@is@a@figure{}} 
\AtEndEnvironment	{table}{\gdef\there@is@a@table{}} 

\AtEndDocument		{\ifdefined\there@is@a@figure\label{fig:was:used:in:doc}\fi}
\AtEndDocument		{\ifdefined\there@is@a@table\label{tab:was:used:in:doc}\fi}

\newcommand{\conditionalLoF}{\@ifundefined{r@fig:was:used:in:doc}{}
	{\addcontentsline{toc}{section}{\listfigurename} \listoffigures \cleardoublepage}}%
\newcommand{\conditionalLoT}{\@ifundefined{r@tab:was:used:in:doc}{}
	{\addcontentsline{toc}{section}{\listtablename} \listoftables \cleardoublepage}}%
\makeatother

% PARAGRAPH SETUP
\newcommand{\dhgeparagraph}[1]{\paragraph{#1}\mbox{}\\\vspace{-1.5em}}

% FOOTNOTE
\renewcommand{\footnotelayout}{\hspace{0.5em}}

% CONSTANTS

    % Projektarbeit Nr. (1 bis 4) oder Bachelorarbeit (B)
    \def\CARBEIT      {1}

    % Title der Arbeit
    \def\CTITLE		    {THEMA}

    % Author der Arbeit (ANR -> Anrede, VOR -> Vorname, NACH -> Nachname)
    \def\CAUTHORANR     {0}     % 1 -> Frau !1 -> Herr
    \def\CAUTHORVOR		{VOR}
    \def\CAUTHORNACH    {NACH}

    % "vorlege am" - Datum
    \def\CDATUM		    {\today}

    % Martikelnummer des Authors
    \def\CMATRIKEL		{MTRNR}

    % Kurs des Auhtors
    \def\CKURS		    {KURS}

    % DHGE Campus des Authors (Gera/Eisenach)
    \def\CCAMPUS		{Gera}

    % Studienbereich des Authors
    \def\CBEREICH		{Technik}

    % Studiengang des Authors
    \def\CSTUDIENGANG	{STUDIENGANG}

    % Betrieb des Authors (nur der Name des Betriebs keine Adresse)
    \def\CBETRIEB		{FIRMA}

    % Betreuer der Arbeit (den akademischen Titel nicht vergessen)
    \def\CBETREUER		{BETREUER}

    % längste Abkürzung in der abk.tex
    \def\CABKL          {DHGE}


% COUNTER (So we can keep counting in Roman after switching to Arabic)
\newcounter{savepage}

%opening
\title{{\LARGE \textbf{\CTITLE}}}
\author			{}
\date			{}

\begin{document}
\pagenumbering	{gobble}
\vspace{\fill}
\maketitle

\if\CARBEIT B

	\begin{center}
		{\LARGE\bf Bachelorarbeit}
		
		\vspace{0.5cm}vorgelegt am \CDATUM
	\end{center}

	\vspace{1cm}

	\def\BETREUER{Gutachter}

\else

	\begin{tabular}{rcccc}
		\hspace{0.45\textwidth} &       I       &     II      &     III     &     IV      \\
	{Projektarbeit Nr.}  \markBox{\CARBEIT}{&}
	\end{tabular}

	\begin{tabular}{rl}
		\hspace{0.45\textwidth} &       \\
		vorgelegt am: & \CDATUM
	\end{tabular}

	\def\BETREUER{Betreuer}

\fi

\begin{tabular}{rl}
	\hspace{0.45\textwidth} &              \\
			von: & \CAUTHOR
\end{tabular}

\begin{tabular}{rl}
	\hspace{0.45\textwidth} &         \\
	Matrikelnummer: & \CMATRIKEL
\end{tabular}

\begin{tabular}{rl}
	\hspace{0.45\textwidth} &      \\
	DHGE Campus: & \CCAMPUS
\end{tabular}

\begin{tabular}{rl}
	\hspace{0.45\textwidth} &         \\
	Studienbereich: & \CBEREICH
\end{tabular}

\begin{tabular}{rl}
	\hspace{0.45\textwidth} &                       \\
	Studiengang: & \CSTUDIENGANG
\end{tabular}

\begin{tabular}{rl}
	\hspace{0.45\textwidth} &       \\
			Kurs: & \CKURS
\end{tabular}

\begin{tabular}{rl}
	\hspace{0.45\textwidth} &          \\
	Ausbildungsstätte: & \CBETRIEB
\end{tabular}

\begin{tabular}{rl}
	\hspace{0.45\textwidth} &          \\
	\BETREUER: & \CBETREUER
\end{tabular}

\vspace*{\fill}

\pagebreak


% INHALTSVERZEICHNIS
\pagenumbering{Roman} \setcounter{page}{1}
\tableofcontents{\fancyfoot{}}
\cleardoublepage

% ABBILDUNGSVERZEICHNIS
%\phantomsection
\conditionalLoF

% TABELLENVERZEICHNIS
%\phantomsection
\conditionalLoT

% ABKÜRZUNGSVERZEICHNIS
\section*{Abkürzungsverzeichnis}
\begin{acronym}[VISBHO]\itemsep0pt
	% Definieren Sie hier Ihre Abkürzungen anhand des DHGE Beispiels.
% Wenn Sie DHGE dann im Text verwenden, rufen sie einfach \ac{dhge} auf.
% LaTeX kümmert sich um den Rest.
% Für alles Weitere schauen Sie sich bitte die Dokumentation des Acronym Packages an.

\acro	{dhge}	[DHGE]		{Duale Hochschule Gera Eisenach}

\end{acronym}
\addcontentsline{toc}{section}{Abkürzungsverzeichnis}
\cleardoublepage

\setcounter{savepage}{\arabic{page}}

% MAIN CONTENT
\pagenumbering	{arabic}
Testtext\footcite{Xarticle} 
\ac{dhge}
\dhgefigure[h]{img}{scale=0.75}{Ein Testbild}{fig:test}[Xarticle][S. 17ff]
\dhgefigure[h]{img}{scale=0.75}{Ein Testbild}{fig:test}
\doubleunderline{$150\mathrm{\Omega}$}
\listofanlagen
\addtoanlagen{fig}{test}
\cleardoublepage

% LITERATURVERZEICHNIS
% TODO Formatierung
\pagenumbering	{Roman} \setcounter{page}{\thesavepage}
\printbibliography[title=Literaturverzeichnis]
\addcontentsline{toc}{section}{Literaturverzeichnis}
\cleardoublepage

%TODO %ANLAGENVERZEICHNIS UND ANLAGEN
\section*{Anlagenverzeichnis}
\addcontentsline{toc}{section}{Anlagen}


\cleardoublepage
% EHRENWÖRTLICHE ERKLÄRUNG
\pagenumbering	{gobble}
\section*{Ehrenwörtliche Erklärung}
\addcontentsline{toc}{section}{Ehrenwörtliche Erklärung}
Ich erkläre hiermit ehrenwörtlich,
\begin{enumerate}[leftmargin=0.5cm]
	\item 	dass ich meine Projektarbeit mit dem Thema:  \\
			\textbf{\CTITLE} \\
			ohne fremde Hilfe angefertigt habe, \\
	\item	dass ich die Übernahme wörtlicher Zitate aus der Literatur sowie die Verwendung der
			Gedanken anderer Autoren an den entsprechenden Stellen innerhalb der Arbeit gekennzeichnet habe und  \\
	\item	dass ich meine Projektarbeit/Studienarbeit/Bachelorarbeit bei keiner anderen Prüfung vorgelegt habe. \\\\
			Ich bin mir bewusst, dass eine falsche Erklärung rechtliche Folgen haben wird. \\\\
\end{enumerate}
\vspace*{\fill}
\begin{tabular} {lrl}
	\hspace{6cm} & \hspace{3cm} & \hspace{6cm} \\
	\hrulefill & & \hrulefill \\
	Ort, Datum & & Unterschrift
\end{tabular}
\vspace*{\fill}


\end{document}
